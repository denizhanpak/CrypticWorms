\documentclass[12pt, a4paper]{article}
\usepackage[margin=0.5in]{geometry}
\usepackage[utf8]{inputenc}
\usepackage[T1]{fontenc}
\usepackage{textcomp}
\usepackage{amsmath, amssymb}


\renewcommand{\baselinestretch}{2}

% figure support
\usepackage{xifthen}
\pdfminorversion=7
\usepackage{pdfpages}
\usepackage{transparent}
\newcommand{\incfig}[1]{%
   \def\svgwidth{\columnwidth}
   \import{./figures/}{#1.pdf_tex}
}

\pdfsuppresswarningpagegroup=1
\title{}
\author{Denizhan Pak}

\begin{document}
\maketitle
\section{Introduction}
Biological systems are characterized by their robust stability under
a wide range of conditions. This is in part achieved by the ability
of such systems to take on different stable states
depending on contextual factors. One set of mechanisms that allows for such 
behavior is the presence of multiple long term stable states. Multistability
appears in a vast range of biological process ranging from microscale
genetic networks to macroscale population interactions. Multistability has
emerged as a useful concept in a range of experimental phenomena.

However, unlike theoretical models in the experimental context stochasticity
is unavoidable. As a result a common empirical practice is to sample what 
would be in the model context, many possible trajectories of the system. 
Such sampling leads to a distribution of the stable states expressed as 
probabilities. Developmental dynamics are a particularly interesting case 
since the distribution of attractor states remaining constant across 
different conditions seems to be a consistent feature.

Two examples of this phenomenon are the regeneration of the planerian flatworm
and the learning of reach-to-grasp in infants. In the case of planerian
theoretical models reveal how under certain environmental conditions 
planerian can develop into two possible configurations. Similarly, infant
prehension can be modeled as bistable with reach without grasp as one state
and reach with grasp as another, expressing both possibilities in the right
environmental context. Although an unexpected experimental finding is that in
both cases: the probability distributions across surprisingly large changes
to the relevant context remains relatively static. 
From a theoretical standpoint this seems 
counter-intuitive since such changes would be expected to shift the basins
of attraction and hence change the resultant probability distribution.

In this work we propose a method to exploit such a regularity. We consider
theoretical models for both phenomena and identify a how such fixed 
probabilities provide a set of constraints on the relationship between model
parameters. This allows us to make predictions as what various parameter 
values should be. We start by demonstrating the method in a simplified model
of bistability  then extend it to models of both flatworm regeneration and
prehension development.

Bistability is abundant in biology (cite). It seems to 
be a fundamental property of processes ranging from multicelularity 
to concscious experience (cite). This recurring theme is appropriate since
bistability is the lowest complexity equilibrium structure aside
from monostability. It is also the nascent mechanism for how a single
sytem can express more than one final state. 

Although bistability has been studied in a range of models, experimental
curioisities still remain. Experimental conditions are probabilistic
with scientific uncertainty playing a fundamental role. In this context
one phenomnon has appeared in multiple developmental systems
and yet remains unexplored in a simpler context. Developmental bistable systems
under experimental conditions seem to have remarkably fixed probability
distributions even as the microscopic conditions of the control
parameters fluctuate.

To explore this phenomnon we propose a method for analyzing the 
relationship between the oberved probability distribution and the control
parameters of bistable model. We first consider the case of a simplified
toy model that will serve as an intuition pump and help us apply the method
to two different experimental systems. We will use a 
model of planarian flatworm regenerataion and model of infant 
prehension learning.

In the cases above we will consider how varying the model parameters
changes the resulting probability distribution and the implications for
those parameters when such a probability distribution is fixed. 

The methods is 6 steps:
\begin{enumerate}
    \item Select a one dimensional model of bistability
    \item Select a sampling distribution (for this work, we consider 
        the case of the normal distribution).
    \item Since the system is one dimensional we can use the cdf of the sampling
        distribution to determine the probability that points are one side
        of the basin of attraction by using the unstable equilibrium as 
        the parameter. This is equivalent to defining a binomial distribution
        from the CDF of a normal distribution.
    \item We set the CDF equal to the observed fixed binomial distribution
        probability
        then identify the correct value for the position of the
        unstable equilibrium point.
    \item Solve for the equillibria as a function of your control parameters
    \item Knowing the correct value for the position of the unstable ep we
        constrain our control parameters using the relation we derived.
        This can give us an experimental prediction for the range of the 
        relevant control parameters.

\end{enumerate}

\section {Results}
\subsection{The Cubic}
Here we build an intuition of the method using the cubic equation.

First we define the dynamical equation governing our toy model. The cubic equation:
\[
    \dot{y} = \beta_1 + \beta_2 y + y^3
\]

The two parameters $\beta_1$ and $\beta_2$ are linearly decomposable and thus
we can visual their influence on the equilibrium points.
\\Visaul here\\

The effect is that $\beta_2$ determines the curvature around the center
and $\beta_1$ determines where the function reaches equilibrium.
If this is within the curved region this is an example of bistability.
We are interested in the case where one samples from the state space of
this system at random. We must make an assumption about the form this sampling
procedure takes. For simplicity we will assume the sampling distribution is a 
normal distribution (mean=0, std=1). Although the method is generic and can
be used for any sampling distribution we will use a normal distribution with
the standard mean and variance for simplicity.

Since this a one-dimensional bistable system whether a given trajectory ends
in one state or the other is simply a function of whether it is greater than
or smaller than the saddle point. This is thus equivalent to asking for the
cumulative distribution function of the sampling distribution CDF(x<X) where
$X$ is the position of the unstable equilibrium point.
\\Visualize this\\
The CDF relates the position of the unstable equilibrium point to the 
binomial probability of falling in one basin or another. In the case of the
normal distribution we write out the cdf:
\\normal cdf\\
However, this is a bit difficult to work with since it requires numerical 
computation. Instead we will use the analytic approximation given by (cite).
\\analytic approximation\\
Now the position of the equilibrium point can uniquely define the resulting
probability of the sampled trajectory entering either basin. To relate this
probability back to the model parameters, we must define the relation from
the parameters $\beta_1$ and $\beta_2$ to the position of the unstable
equilibrium point. In the case of the cubic we can do this analytically.
We relate the two as follows:
\[
    \dot{y} = \beta_1 + \beta_2 y + y^3
    \rightarrow \beta_1-\dot{y}=-\beta_2 y + y^3
    \rightarrow \beta_1 + \beta_2 y -\dot{y}=y^3
    \rightarrow \beta_1+\beta_2 y= y^3 + \dot{y}
\]
We have the additional constraint that
\[
    \ddot{y} > 0
\]
by virtue of the fact that we are looking for the unstable equilibrium point.
Therefore by specifying a given probability we can identify an $X$ which
specifies a position for the unstable equilibrium. If we plot the position
of the equilibrium points as a function of the parameters $\beta_1$ and $\beta_2$,
we get the familiar looking cusp bifurcation. We can embed that cusp bifurcation
in a 3D space by adding the probability distribution defined by that point in 
parameter space as the third dimension
\\viz here\\
Given that visualization we can simply draw a plane at the desired probability point.
This defines a line or a relation between $\beta_1$ and $\beta_2$ that must be
preserved for the system to have the given probability that it does. 

\end{document}
